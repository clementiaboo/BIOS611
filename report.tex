% Options for packages loaded elsewhere
\PassOptionsToPackage{unicode}{hyperref}
\PassOptionsToPackage{hyphens}{url}
%
\documentclass[
]{article}
\usepackage{amsmath,amssymb}
\usepackage{lmodern}
\usepackage{ifxetex,ifluatex}
\ifnum 0\ifxetex 1\fi\ifluatex 1\fi=0 % if pdftex
  \usepackage[T1]{fontenc}
  \usepackage[utf8]{inputenc}
  \usepackage{textcomp} % provide euro and other symbols
\else % if luatex or xetex
  \usepackage{unicode-math}
  \defaultfontfeatures{Scale=MatchLowercase}
  \defaultfontfeatures[\rmfamily]{Ligatures=TeX,Scale=1}
\fi
% Use upquote if available, for straight quotes in verbatim environments
\IfFileExists{upquote.sty}{\usepackage{upquote}}{}
\IfFileExists{microtype.sty}{% use microtype if available
  \usepackage[]{microtype}
  \UseMicrotypeSet[protrusion]{basicmath} % disable protrusion for tt fonts
}{}
\makeatletter
\@ifundefined{KOMAClassName}{% if non-KOMA class
  \IfFileExists{parskip.sty}{%
    \usepackage{parskip}
  }{% else
    \setlength{\parindent}{0pt}
    \setlength{\parskip}{6pt plus 2pt minus 1pt}}
}{% if KOMA class
  \KOMAoptions{parskip=half}}
\makeatother
\usepackage{xcolor}
\IfFileExists{xurl.sty}{\usepackage{xurl}}{} % add URL line breaks if available
\IfFileExists{bookmark.sty}{\usepackage{bookmark}}{\usepackage{hyperref}}
\hypersetup{
  pdftitle={test report},
  pdfauthor={Clement Li},
  hidelinks,
  pdfcreator={LaTeX via pandoc}}
\urlstyle{same} % disable monospaced font for URLs
\usepackage[margin=1in]{geometry}
\usepackage{graphicx}
\makeatletter
\def\maxwidth{\ifdim\Gin@nat@width>\linewidth\linewidth\else\Gin@nat@width\fi}
\def\maxheight{\ifdim\Gin@nat@height>\textheight\textheight\else\Gin@nat@height\fi}
\makeatother
% Scale images if necessary, so that they will not overflow the page
% margins by default, and it is still possible to overwrite the defaults
% using explicit options in \includegraphics[width, height, ...]{}
\setkeys{Gin}{width=\maxwidth,height=\maxheight,keepaspectratio}
% Set default figure placement to htbp
\makeatletter
\def\fps@figure{htbp}
\makeatother
\setlength{\emergencystretch}{3em} % prevent overfull lines
\providecommand{\tightlist}{%
  \setlength{\itemsep}{0pt}\setlength{\parskip}{0pt}}
\setcounter{secnumdepth}{-\maxdimen} % remove section numbering
\ifluatex
  \usepackage{selnolig}  % disable illegal ligatures
\fi

\title{test report}
\author{Clement Li}
\date{11/29/2021}

\begin{document}
\maketitle

\hypertarget{introduction}{%
\subsubsection{Introduction}\label{introduction}}

Grades and test scores---these are, in nearly all contemporary academic
settings, the sole representations of student success. The focus of my
analysis is to delve into the many relationships between student
information and school performance and extract those characteristics
that seem to have the greatest impact on a student's performance in
school.

\hypertarget{data}{%
\subsubsection{Data}\label{data}}

This dataset is a sample of high school students obtained through a
survey of students taking Math and Portuguese language courses at two
high schools in Portugal. Each observation represents the survey
response of a single student; the variables gathered from the survey
contain a wide breadth of information about the students and their
families --- in total, there are 33 variables.

I split the variables into four categories: response, behavioral,
background, and educational. The response variables are G4 and G3. G4 is
a binary variable we created based on the given variables G1 and G2,
which are the grades for the first and second term; G4 is 1 if G2 is
greater than or equal to G1 (indicating improvement from the first term
grade) and 0 otherwise. G3 is a numeric variable representing the final
grade of a student, given in points out of 20. G1 and G2, mentioned
earlier, use the same units as G3.

The behavioral variables are Alc, goout, freetime, romantic, activities,
and health. Alc, defined as 1 if the student drinks and 0 if the student
does not drink, is a binary variable we created using two other
variables, Walc and Dalc (weekend and weekday alcohol consumption). The
factor variable goout describes the amount of time a student spends
going out with friends (``Very Low'' to ``Very High''). Freetime
represents the amount of free time a student has after school (also
``Very Low'' to ``Very High''). Romantic is a binary variable indicating
whether a student is in a romantic relationship (1) or not (0).
Similarly, activities is a binary variable that is 1 if a student
participates in extracurricular activities. Finally, health is a factor
variable with 5 levels that describes the overall health status of a
student (``Very Bad'' to ``Very Good'').

The background variables are sex, address, Mjob, Pjob, nursery, Gedu,
famsize, internet, and traveltime. Sex is a binary variable with value
``F'' for female and ``M'' for male. Address is also a binary variable
indicating whether a student lives in an ``Urban'' or ``Rural'' area.
Mjob and Fjob are factor variables representing the occupation of a
student's mother and father, respectively (``teacher,'' ``health''
related, civil ``services,'' ``at\_home,'' and ``other''). Nursery is a
binary variable that is 1 if the student attended nursery school as a
child and 0 otherwise. Internet represents whether a student does (1) or
does not (0) have home Internet access. Gedu is a categorical metric we
created based on three other variables, guardian, Medu, and Fedu, that
denotes the highest completed education level of a student's guardian
(``Minimal'' education, ``Middle'' school, ``Secondary'' school, and
``Higher'' education). Traveltime is a numeric variable representing the
time, in hours, a student spends going from home to school.

The educational variables are schoolsup, famsup, paid, absences,
studytime, failures, firstgen, and reason. Schoolsup and famsup are
binary variables indicating whether a student receives school and family
educational support, respectively. Paid is another binary variable that
reveals whether a student takes paid classes outside of school. Absences
is a discrete numeric variable denoting the number of school absences.
Failures is also a discrete numeric variable counting the number of past
class failures. Reason describes the reason why a student chose to
attend their high school: proximity to ``home,'' school ``reputation,''
``course'' preference, and ``other.'' Firstgen is a binary variable we
made using Medu, Fedu, and higher (a variable representing whether a
student wants to pursue higher education); firstgen indicates whether a
student is an aspiring first-generation college student, defined as a
student who wants to go to college (based on higher) and whose parents
both never completed secondary school. We do not have enough information
to confirm whether these students match the traditional definition of
first-generation but named the variable firstgen merely out of
convenience.

\begin{figure}
\centering
\includegraphics{./figures/f1.png}
\caption{caption}
\end{figure}

My final project seeks to answer the question: ``which educational
variables act as the best predictors for a student's final grade?'' I
trained several different regression models, each using a different
subset of variables and their two-way interactions. The models were
first trained on the same 549 observation training set and compared on
the basis of their performances on the 100 observation test set.

The first model I tested is the most basic linear model that simply uses
each predictor in the regression; this model was named ``full.'' From
this model, I applied two different variable selection methods: backward
selection and best subsets. For the simple multiple regression model,
both backward selection and best subsets resulted in the same model that
we called ``Best,'' which uses schoolsup, studytime, failures, reason,
firstgen as predictors.

In addition, I accounted for two-way interactions and built models using
a similar variable selection process. I first created Full2, which
considers all predictors as well as all the two-way interaction terms.
Then, I used backward selection and the best subsets methods; in this
case, these resulted in two different models, which are labeled as
``Back2'' and ``Best2.''

I also trained lasso and ridge regression models using all the
educational variables and their two-way interactions and chose the
optimal lasso and ridge penalty parameters with 10-fold
cross-validation.

\begin{figure}
\centering
\includegraphics{./figures/figure2.png}
\caption{caption}
\end{figure}

Once all the models were trained, I evaluated their performances on the
test set using RMSE (root-mean-square error). Lasso and Best2 had the
two lowest RMSE values, and their test performances were quite similar.
The lasso regression model uses just 2 predictors: studytime and
failures. Best2 uses several more predictors: studytime, failures,
reason, studytime:schoolsup, failures:schoolsup, reason:famsup,
studytime:reason, failures:absences.

Though the Best2 model uses over triple the amount of predictors, the
lasso regression model gives a very similar test performance, which
leads us to suspect that those other variables are more ``noise'' than
``signal.''

In conclusion, the two variables most predictive of G3 (final grade) are
studytime and failures (number of previous classes failed). The plots
below demonstrate the strength of the relationship between studytime and
G3 as well as failures and G3.

\begin{figure}
\centering
\includegraphics{./figures/figure3.png}
\caption{caption}
\end{figure}

\end{document}
